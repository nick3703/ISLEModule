% Options for packages loaded elsewhere
\PassOptionsToPackage{unicode}{hyperref}
\PassOptionsToPackage{hyphens}{url}
%
\documentclass[
  ignorenonframetext,
]{beamer}
\usepackage{pgfpages}
\setbeamertemplate{caption}[numbered]
\setbeamertemplate{caption label separator}{: }
\setbeamercolor{caption name}{fg=normal text.fg}
\beamertemplatenavigationsymbolsempty
% Prevent slide breaks in the middle of a paragraph
\widowpenalties 1 10000
\raggedbottom
\setbeamertemplate{part page}{
  \centering
  \begin{beamercolorbox}[sep=16pt,center]{part title}
    \usebeamerfont{part title}\insertpart\par
  \end{beamercolorbox}
}
\setbeamertemplate{section page}{
  \centering
  \begin{beamercolorbox}[sep=12pt,center]{part title}
    \usebeamerfont{section title}\insertsection\par
  \end{beamercolorbox}
}
\setbeamertemplate{subsection page}{
  \centering
  \begin{beamercolorbox}[sep=8pt,center]{part title}
    \usebeamerfont{subsection title}\insertsubsection\par
  \end{beamercolorbox}
}
\AtBeginPart{
  \frame{\partpage}
}
\AtBeginSection{
  \ifbibliography
  \else
    \frame{\sectionpage}
  \fi
}
\AtBeginSubsection{
  \frame{\subsectionpage}
}
\usepackage{amsmath,amssymb}
\usepackage{lmodern}
\usepackage{iftex}
\ifPDFTeX
  \usepackage[T1]{fontenc}
  \usepackage[utf8]{inputenc}
  \usepackage{textcomp} % provide euro and other symbols
\else % if luatex or xetex
  \usepackage{unicode-math}
  \defaultfontfeatures{Scale=MatchLowercase}
  \defaultfontfeatures[\rmfamily]{Ligatures=TeX,Scale=1}
\fi
% Use upquote if available, for straight quotes in verbatim environments
\IfFileExists{upquote.sty}{\usepackage{upquote}}{}
\IfFileExists{microtype.sty}{% use microtype if available
  \usepackage[]{microtype}
  \UseMicrotypeSet[protrusion]{basicmath} % disable protrusion for tt fonts
}{}
\makeatletter
\@ifundefined{KOMAClassName}{% if non-KOMA class
  \IfFileExists{parskip.sty}{%
    \usepackage{parskip}
  }{% else
    \setlength{\parindent}{0pt}
    \setlength{\parskip}{6pt plus 2pt minus 1pt}}
}{% if KOMA class
  \KOMAoptions{parskip=half}}
\makeatother
\usepackage{xcolor}
\IfFileExists{xurl.sty}{\usepackage{xurl}}{} % add URL line breaks if available
\IfFileExists{bookmark.sty}{\usepackage{bookmark}}{\usepackage{hyperref}}
\hypersetup{
  pdftitle={Is a sub 2 hour marathon in the near future? Modeling rare events in sports.},
  pdfauthor={Rodney X. Sturdivant, Ph.D., Baylor University and Nick Clark, Ph.D., West Point},
  hidelinks,
  pdfcreator={LaTeX via pandoc}}
\urlstyle{same} % disable monospaced font for URLs
\newif\ifbibliography
\usepackage{longtable,booktabs,array}
\usepackage{calc} % for calculating minipage widths
\usepackage{caption}
% Make caption package work with longtable
\makeatletter
\def\fnum@table{\tablename~\thetable}
\makeatother
\usepackage{graphicx}
\makeatletter
\def\maxwidth{\ifdim\Gin@nat@width>\linewidth\linewidth\else\Gin@nat@width\fi}
\def\maxheight{\ifdim\Gin@nat@height>\textheight\textheight\else\Gin@nat@height\fi}
\makeatother
% Scale images if necessary, so that they will not overflow the page
% margins by default, and it is still possible to overwrite the defaults
% using explicit options in \includegraphics[width, height, ...]{}
\setkeys{Gin}{width=\maxwidth,height=\maxheight,keepaspectratio}
% Set default figure placement to htbp
\makeatletter
\def\fps@figure{htbp}
\makeatother
\setlength{\emergencystretch}{3em} % prevent overfull lines
\providecommand{\tightlist}{%
  \setlength{\itemsep}{0pt}\setlength{\parskip}{0pt}}
\setcounter{secnumdepth}{-\maxdimen} % remove section numbering
\ifLuaTeX
  \usepackage{selnolig}  % disable illegal ligatures
\fi

\title{Is a sub 2 hour marathon in the near future? Modeling rare events
in sports.}
\author{Rodney X. Sturdivant, Ph.D., Baylor University and Nick Clark,
Ph.D., West Point}
\date{}

\begin{document}
\frame{\titlepage}

\begin{frame}{Outline}
\protect\hypertarget{outline}{}
\begin{itemize}
\item
  Baseball Rare Events (if needed only)
\item
  Background
\item
  Marathon Data
\item
  Simple Model
\item
  Self-Exciting Model
\item
  Further Research
\end{itemize}
\end{frame}

\begin{frame}{Background}
\protect\hypertarget{background}{}
\begin{block}{Are we living in a time of records?}
\protect\hypertarget{are-we-living-in-a-time-of-records}{}
\begin{itemize}
\tightlist
\item
  Include NY Times article screenshot of headline.
\item
  Brief summary of article's premise.
\end{itemize}
\end{block}

\begin{block}{How can we address this question?}
\protect\hypertarget{how-can-we-address-this-question}{}
\end{block}

\begin{block}{What would randomness look like?}
\protect\hypertarget{what-would-randomness-look-like}{}
Pictures of Rod and Nick running
\end{block}
\end{frame}

\begin{frame}{Marathon World Record Data}
\protect\hypertarget{marathon-world-record-data}{}
Men's Marathon world records since 1908

NEED TO CLEAN UP - NICER TABLE WITH JUST TIME NAME NATIONALITY DATE
MAYBE INCLUDE A COUPLE OF PICTURES OF PEOPLE

\begin{longtable}[]{@{}
  >{\raggedright\arraybackslash}p{(\columnwidth - 20\tabcolsep) * \real{0.0229}}
  >{\raggedright\arraybackslash}p{(\columnwidth - 20\tabcolsep) * \real{0.0435}}
  >{\raggedright\arraybackslash}p{(\columnwidth - 20\tabcolsep) * \real{0.0343}}
  >{\raggedright\arraybackslash}p{(\columnwidth - 20\tabcolsep) * \real{0.0435}}
  >{\raggedright\arraybackslash}p{(\columnwidth - 20\tabcolsep) * \real{0.1030}}
  >{\raggedright\arraybackslash}p{(\columnwidth - 20\tabcolsep) * \real{0.0526}}
  >{\raggedright\arraybackslash}p{(\columnwidth - 20\tabcolsep) * \real{0.5995}}
  >{\raggedleft\arraybackslash}p{(\columnwidth - 20\tabcolsep) * \real{0.0297}}
  >{\raggedleft\arraybackslash}p{(\columnwidth - 20\tabcolsep) * \real{0.0206}}
  >{\raggedright\arraybackslash}p{(\columnwidth - 20\tabcolsep) * \real{0.0252}}
  >{\raggedright\arraybackslash}p{(\columnwidth - 20\tabcolsep) * \real{0.0252}}@{}}
\toprule
\begin{minipage}[b]{\linewidth}\raggedright
Time
\end{minipage} & \begin{minipage}[b]{\linewidth}\raggedright
Name
\end{minipage} & \begin{minipage}[b]{\linewidth}\raggedright
Nationality
\end{minipage} & \begin{minipage}[b]{\linewidth}\raggedright
Date
\end{minipage} & \begin{minipage}[b]{\linewidth}\raggedright
Event/Place
\end{minipage} & \begin{minipage}[b]{\linewidth}\raggedright
Source
\end{minipage} & \begin{minipage}[b]{\linewidth}\raggedright
Notes
\end{minipage} & \begin{minipage}[b]{\linewidth}\raggedleft
Time\_t
\end{minipage} & \begin{minipage}[b]{\linewidth}\raggedleft
Time\_sec
\end{minipage} & \begin{minipage}[b]{\linewidth}\raggedright
Date\_ymd
\end{minipage} & \begin{minipage}[b]{\linewidth}\raggedright
end
\end{minipage} \\
\midrule
\endhead
2:55:18.4 & Johnny Hayes & United States & July 24, 1908 & London,
United Kingdom & IAAF{[}53{]} & Time was officially recorded as 2:55:18
2/5.{[}54{]}Italian Dorando Pietri finished in 2:54:46.4, but was
disqualified for receiving assistance from race officials near the
finish.{[}55{]} Note.{[}56{]} & 2H 55M 18.4S & 10518.4 & 1908-07-24 &
1909-01-01 \\
2:52:45.4 & Robert Fowler & United States & January 1, 1909 &
Yonkers,{[}nb 5{]}United States & IAAF{[}53{]} & Note.{[}56{]} & 2H 52M
45.4S & 10365.4 & 1909-01-01 & 1909-02-12 \\
2:46:52.8 & James Clark & United States & February 12, 1909 & New York
City, United States & IAAF{[}53{]} & Note.{[}56{]} & 2H 46M 52.8S &
10012.8 & 1909-02-12 & 1909-05-08 \\
2:46:04.6 & Albert Raines & United States & May 8, 1909 & New York City,
United States & IAAF{[}53{]} & Note.{[}56{]} & 2H 46M 4.6S & 9964.6 &
1909-05-08 & 1909-05-10 \\
2:42:31.0 & Henry Barrett & United Kingdom & May 8, 1909{[}nb 6{]} &
Polytechnic Marathon, London, United Kingdom & IAAF{[}53{]} &
Note.{[}56{]} & 2H 42M 31S & 9751.0 & 1909-05-10 & 1909-08-31 \\
2:40:34.2 & Thure Johansson & Sweden & August 31, 1909 & Stockholm,
Sweden & IAAF{[}53{]} & Note.{[}56{]} & 2H 40M 34.2S & 9634.2 &
1909-08-31 & 1913-05-12 \\
2:38:16.2 & Harry Green & United Kingdom & May 12, 1913 & Polytechnic
Marathon & IAAF{[}53{]} & Note.{[}61{]} & 2H 38M 16.2S & 9496.2 &
1913-05-12 & 1913-05-31 \\
2:36:06.6 & Alexis Ahlgren & Sweden & May 31, 1913 & Polytechnic
Marathon & IAAF{[}53{]} & Report in The Times claiming world
record.{[}62{]} Note.{[}61{]} & 2H 36M 6.6S & 9366.6 & 1913-05-31 &
1914-11-29 \\
2:38:00.8 & Umberto Blasi & Italy & November 29, 1914 & Legnano, Italy &
ARRS{[}9{]} & & 2H 38M 0.8S & 9480.8 & 1914-11-29 & 1920-08-22 \\
2:32:35.8 & Hannes Kolehmainen & Finland & August 22, 1920 & Antwerp,
Belgium & IAAF,{[}53{]} ARRS{[}9{]} & The course distance was officially
reported to be 42,750 meters/26.56 miles,{[}63{]} however, the
Association of Road Racing Statisticians estimated the course to be
40~km.{[}31{]} & 2H 32M 35.8S & 9155.8 & 1920-08-22 & 1925-10-12 \\
2:29:01.8 & Albert Michelsen & United States & October 12, 1925 & Port
Chester, United States & IAAF{[}53{]} & Note.{[}64{]} & 2H 29M 1.8S &
8941.8 & 1925-10-12 & 1929-07-05 \\
2:30:57.6 & Harry Payne & United Kingdom & July 5, 1929 & London &
ARRS{[}9{]} & & 2H 30M 57.6S & 9057.6 & 1929-07-05 & 1935-03-21 \\
2:26:14 & Sohn Kee-chung & Japanese Korea & March 21, 1935 & Tokyo,
Japan & ARRS{[}9{]} & Competed for Japan as Kitei Son because of Japan's
occupation of the Korean Peninsula & 2H 26M 14S & 8774.0 & 1935-03-21 &
1935-03-31 \\
2:27:49.0 & Fusashige Suzuki & Japan & March 31, 1935 & Tokyo, Japan &
IAAF{[}53{]} & According to the Association of Road Racing
Statisticians, Suzuki's 2:27:49 performance occurred in Tokyo on March
21, 1935, during a race in which he finished second to Sohn Kee-chung
(sometimes referred to as Kee-Jung Sohn or Son Kitei) who ran a
2:26:14.{[}65{]} & 2H 27M 49S & 8869.0 & 1935-03-31 & 1935-04-03 \\
2:26:44.0 & Yasuo Ikenaka & Japan & April 3, 1935 & Tokyo, Japan &
IAAF{[}53{]} & Note.{[}66{]} & 2H 26M 44S & 8804.0 & 1935-04-03 &
1935-11-03 \\
2:26:42 & Sohn Kee-chung & Japanese Korea & November 3, 1935 & Tokyo,
Japan & IAAF{[}53{]} & Note.{[}66{]} & 2H 26M 42S & 8802.0 & 1935-11-03
& 1947-04-19 \\
2:25:39 & Suh Yun-bok & Korea & April 19, 1947 & Boston Marathon &
IAAF{[}53{]} & Disputed (short course).{[}67{]} Disputed
(point-to-point).{[}68{]} Note.{[}69{]} & 2H 25M 39S & 8739.0 &
1947-04-19 & 1952-06-14 \\
2:20:42.2 & Jim Peters & United Kingdom & June 14, 1952 & Polytechnic
Marathon & IAAF,{[}53{]} ARRS{[}9{]} & MarathonGuide.com states the
course was slightly long.{[}70{]} Report in The Times claiming world
record.{[}71{]} & 2H 20M 42.2S & 8442.2 & 1952-06-14 & 1953-06-13 \\
2:18:40.4 & Jim Peters & United Kingdom & June 13, 1953 & Polytechnic
Marathon & IAAF,{[}53{]} ARRS{[}9{]} & Report in The Times claiming
world record.{[}71{]} & 2H 18M 40.4S & 8320.4 & 1953-06-13 &
1953-10-04 \\
2:18:34.8 & Jim Peters & United Kingdom & October 4, 1953 & Turku
Marathon & IAAF,{[}53{]} ARRS{[}9{]} & & 2H 18M 34.8S & 8314.8 &
1953-10-04 & 1954-06-26 \\
2:17:39.4 & Jim Peters & United Kingdom & June 26, 1954 & Polytechnic
Marathon & IAAF{[}53{]} & Point-to-point course.{[}citation needed{]}
Report in The Times claiming world record.{[}72{]} & 2H 17M 39.4S &
8259.4 & 1954-06-26 & 1956-08-12 \\
2:18:04.8 & Paavo Kotila & Finland & August 12, 1956 & Pieksämäki,
Finland & ARRS{[}9{]} & & 2H 18M 4.8S & 8284.8 & 1956-08-12 &
1958-08-24 \\
2:15:17.0 & Sergei Popov & Soviet Union & August 24, 1958 & Stockholm,
Sweden & IAAF,{[}53{]} ARRS{[}9{]} & The ARRS notes Popov's extended
time as 2:15:17.6{[}9{]} & 2H 15M 17S & 8117.0 & 1958-08-24 &
1960-09-10 \\
2:15:16.2 & Abebe Bikila & Ethiopia & September 10, 1960 & Rome, Italy &
IAAF,{[}53{]} ARRS{[}9{]} & World record fastest marathon run in bare
feet.{[}73{]} & 2H 15M 16.2S & 8116.2 & 1960-09-10 & 1963-02-17 \\
2:15:15.8 & Toru Terasawa & Japan & February 17, 1963 & Beppu-Ōita
Marathon & IAAF,{[}53{]} ARRS{[}9{]} & & 2H 15M 15.8S & 8115.8 &
1963-02-17 & 1963-06-15 \\
2:14:28 & Leonard Edelen & United States & June 15, 1963 & Polytechnic
Marathon & IAAF{[}53{]} & Point-to-point course.{[}citation needed{]}
Report in The Times claiming world record and stating that the course
may have been long.{[}74{]} & 2H 14M 28S & 8068.0 & 1963-06-15 &
1963-07-06 \\
2:14:43 & Brian Kilby & United Kingdom & July 6, 1963 & Port Talbot,
Wales & ARRS{[}9{]} & & 2H 14M 43S & 8083.0 & 1963-07-06 & 1964-06-13 \\
2:13:55 & Basil Heatley & United Kingdom & June 13, 1964 & Polytechnic
Marathon & IAAF{[}53{]} & Point-to-point course.{[}citation needed{]}
Report in The Times claiming world record.{[}75{]} & 2H 13M 55S & 8035.0
& 1964-06-13 & 1964-10-21 \\
2:12:11.2 & Abebe Bikila & Ethiopia & October 21, 1964 & Tokyo, Japan &
IAAF,{[}53{]} ARRS{[}9{]} & & 2H 12M 11.2S & 7931.2 & 1964-10-21 &
1965-06-12 \\
2:12:00 & Morio Shigematsu & Japan & June 12, 1965 & Polytechnic
Marathon & IAAF{[}53{]} & Point-to-point course.{[}citation needed{]}
Report in The Times claiming world record.{[}76{]} & 2H 12M 0S & 7920.0
& 1965-06-12 & 1967-12-03 \\
2:09:36.4 & Derek Clayton & Australia & December 3, 1967 & Fukuoka
Marathon & IAAF,{[}53{]} ARRS{[}9{]} & & 2H 9M 36.4S & 7776.4 &
1967-12-03 & 1969-05-30 \\
2:08:33.6 & Derek Clayton & Australia & May 30, 1969 & Antwerp, Belgium
& IAAF{[}53{]} & Disputed (short course).{[}77{]} & 2H 8M 33.6S & 7713.6
& 1969-05-30 & 1970-07-23 \\
2:09:28.8 & Ron Hill & United Kingdom & July 23, 1970 & Edinburgh,
Scotland & ARRS{[}9{]} & & 2H 9M 28.8S & 7768.8 & 1970-07-23 &
1974-01-31 \\
2:09:12 & Ian Thompson & United Kingdom & January 31, 1974 &
Christchurch, New Zealand & ARRS{[}9{]} & & 2H 9M 12S & 7752.0 &
1974-01-31 & 1978-02-05 \\
2:09:05.6 & Shigeru So & Japan & February 5, 1978 & Beppu-Ōita Marathon
& ARRS{[}9{]} & & 2H 9M 5.6S & 7745.6 & 1978-02-05 & 1980-04-26 \\
2:09:01 & Gerard Nijboer & Netherlands & April 26, 1980 & Amsterdam
Marathon & ARRS{[}9{]} & & 2H 9M 1S & 7741.0 & 1980-04-26 &
1981-12-06 \\
2:08:18 & Robert De Castella & Australia & December 6, 1981 & Fukuoka
Marathon & IAAF,{[}53{]} ARRS{[}9{]} & & 2H 8M 18S & 7698.0 & 1981-12-06
& 1984-10-21 \\
2:08:05 & Steve Jones & United Kingdom & October 21, 1984 & Chicago
Marathon & IAAF,{[}53{]} ARRS{[}9{]} & & 2H 8M 5S & 7685.0 & 1984-10-21
& 1985-04-20 \\
2:07:12 & Carlos Lopes & Portugal & April 20, 1985 & Rotterdam Marathon
& IAAF,{[}53{]} ARRS{[}9{]} & & 2H 7M 12S & 7632.0 & 1985-04-20 &
1988-04-17 \\
2:06:50 & Belayneh Dinsamo & Ethiopia & April 17, 1988 & Rotterdam
Marathon & IAAF,{[}53{]} ARRS{[}9{]} & & 2H 6M 50S & 7610.0 & 1988-04-17
& 1998-09-20 \\
2:06:05 & Ronaldo da Costa & Brazil & September 20, 1998 & Berlin
Marathon & IAAF,{[}53{]} ARRS{[}9{]} & & 2H 6M 5S & 7565.0 & 1998-09-20
& 1999-10-24 \\
2:05:42 & Khalid Khannouchi & Morocco & October 24, 1999 & Chicago
Marathon & IAAF,{[}53{]} ARRS{[}9{]} & & 2H 5M 42S & 7542.0 & 1999-10-24
& 2002-04-14 \\
2:05:38 & Khalid Khannouchi & United States & April 14, 2002 & London
Marathon & IAAF,{[}53{]} ARRS{[}9{]} & First ``World's Best'' recognized
by the International Association of Athletics Federations.{[}78{]} The
ARRS notes Khannouchi's extended time as 2:05:37.8{[}9{]} & 2H 5M 38S &
7538.0 & 2002-04-14 & 2003-09-28 \\
2:04:55 & Paul Tergat & Kenya & September 28, 2003 & Berlin Marathon &
IAAF,{[}53{]} ARRS{[}9{]} & First world record for the men's marathon
ratified by the International Association of Athletics
Federations.{[}79{]} & 2H 4M 55S & 7495.0 & 2003-09-28 & 2007-09-30 \\
2:04:26 & Haile Gebrselassie & Ethiopia & September 30, 2007 & Berlin
Marathon & IAAF,{[}53{]} ARRS{[}9{]} & & 2H 4M 26S & 7466.0 & 2007-09-30
& 2008-09-28 \\
2:03:59 & Haile Gebrselassie & Ethiopia & September 28, 2008 & Berlin
Marathon & IAAF,{[}53{]} ARRS{[}9{]} & The ARRS notes Gebrselassie's
extended time as 2:03:58.2{[}9{]} & 2H 3M 59S & 7439.0 & 2008-09-28 &
2011-09-25 \\
2:03:38 & Patrick Makau & Kenya & September 25, 2011 & Berlin Marathon &
IAAF,{[}80{]}{[}81{]} ARRS{[}82{]} & & 2H 3M 38S & 7418.0 & 2011-09-25 &
2013-09-29 \\
2:03:23 & Wilson Kipsang & Kenya & September 29, 2013 & Berlin Marathon
& IAAF{[}83{]}{[}84{]} ARRS{[}82{]} & The ARRS notes Kipsang's extended
time as 2:03:22.2{[}82{]} & 2H 3M 23S & 7403.0 & 2013-09-29 &
2014-09-28 \\
2:02:57 & Dennis Kimetto & Kenya & September 28, 2014 & Berlin Marathon
& IAAF{[}85{]}{[}86{]} ARRS{[}82{]} & The ARRS notes Kimetto's extended
time as 2:02:56.4{[}82{]} & 2H 2M 57S & 7377.0 & 2014-09-28 &
2018-09-16 \\
2:01:39 & Eliud Kipchoge & Kenya & September 16, 2018 & Berlin Marathon
& IAAF{[}1{]} & & 2H 1M 39S & 7299.0 & 2018-09-16 & 2022-08-01 \\
\bottomrule
\end{longtable}
\end{frame}

\begin{frame}{Visualizing the data}
\protect\hypertarget{visualizing-the-data}{}
AGAIN NEED CLEANING UP - NEED TIMES IN SOMETHING OTHER THAN SECONDS
MAYBE INCLUDE 2 HOUR HORIZONTAL BAR MAYBE HAVE A SLIDE WHERE ADD
PICTURES OF PEOPLE WHO LOWERED RECORD BY A LOT WHICH PLOT(S) TO USE?
MAYBE TWO OF THEM BUT ON ONE SLIDE?

\includegraphics{sturdivant_clark_srcos2022_files/figure-beamer/unnamed-chunk-2-1.pdf}
\end{frame}

\begin{frame}{Visualize B}
\protect\hypertarget{visualize-b}{}
\includegraphics{sturdivant_clark_srcos2022_files/figure-beamer/unnamed-chunk-3-1.pdf}
\end{frame}

\begin{frame}{Visualize C}
\protect\hypertarget{visualize-c}{}
\includegraphics{sturdivant_clark_srcos2022_files/figure-beamer/unnamed-chunk-4-1.pdf}
\end{frame}

\begin{frame}{SIMPLE MODEL}
\protect\hypertarget{simple-model}{}
\begin{block}{POISSON PROCESS (NEED TO SHORTEN, OR PUT ON TWO SLIDES)}
\protect\hypertarget{poisson-process-need-to-shorten-or-put-on-two-slides}{}
A model for a series of discrete events where the average time between
events is known, but the exact timing of events is ``random'' meeting
the following criteria:

\begin{itemize}
\item
  Events are independent of each other. The occurrence of one event does
  not affect the probability another event will occur.
\item
  The average rate (events per time period) is constant.
\item
  Two events cannot occur at the same time.
\end{itemize}

The time between events (known as the interarrival times) follow an
exponential distribution defined as:

\[P(T>t) = e^{-\lambda t}\] Where T is the random variable of the time
until the next event, t is a specific time for the next event, and
\(\lambda\) is the rate: the average number of events per unit of time.
Note the possible values of T are greater than 0 (positive only).
\end{block}
\end{frame}

\begin{frame}[fragile]{Reasonableness of Exponential Interarrivals}
\protect\hypertarget{reasonableness-of-exponential-interarrivals}{}
The exponential distribution has certain attributes, for example:

\(E(T) = 1/\lambda\) \(SD(T) = 1/\lambda\)

The mean and standard deviation of the years between records:

\begin{verbatim}
## [1] 2.249427
\end{verbatim}

\begin{verbatim}
## [1] 2.428191
\end{verbatim}

\includegraphics{sturdivant_clark_srcos2022_files/figure-beamer/unnamed-chunk-5-1.pdf}
\end{frame}

\begin{frame}[fragile]{MORE ON THE SIMPLE MODEL}
\protect\hypertarget{more-on-the-simple-model}{}
We estimate (MLE) \(\lambda = 1/E(T)\)

\begin{verbatim}
##      rate 
## 0.4445577
\end{verbatim}

Model fit

\includegraphics{sturdivant_clark_srcos2022_files/figure-beamer/unnamed-chunk-7-1.pdf}
\end{frame}

\begin{frame}[fragile]{Fit of simple model}
\protect\hypertarget{fit-of-simple-model}{}
\begin{verbatim}
## 
##  Asymptotic one-sample Kolmogorov-Smirnov test
## 
## data:  days_between_mod2
## D = 0.078053, p-value = 0.9264
## alternative hypothesis: two-sided
\end{verbatim}

\begin{verbatim}
## 
##  Cramer-von Mises test of goodness-of-fit
##  Braun's adjustment using 7 groups
##  Null hypothesis: exponential distribution
##  with parameter rate = 0.444557679401457
##  Parameters assumed to have been estimated from data
## 
## data:  days_between_mod2
## omega2max = 0.32886, p-value = 0.56
\end{verbatim}

\begin{verbatim}
## 
##  Anderson-Darling test of goodness-of-fit
##  Braun's adjustment using 7 groups
##  Null hypothesis: exponential distribution
##  with parameter rate = 0.444557679401457
##  Parameters assumed to have been estimated from data
## 
## data:  days_between_mod2
## Anmax = 2.147, p-value = 0.4345
\end{verbatim}
\end{frame}

\begin{frame}{Fit of simple model B}
\protect\hypertarget{fit-of-simple-model-b}{}
\includegraphics{sturdivant_clark_srcos2022_files/figure-beamer/unnamed-chunk-9-1.pdf}
\end{frame}

\begin{frame}{Are records then random?}
\protect\hypertarget{are-records-then-random}{}
\includegraphics{sturdivant_clark_srcos2022_files/figure-beamer/unnamed-chunk-10-1.pdf}
\end{frame}

\begin{frame}{What are the poorly fit points?}
\protect\hypertarget{what-are-the-poorly-fit-points}{}
LOOK BACK AT THE ORIGINAL DATA HERE\ldots LONGEST TIMES BETWEEN EVENTS
(I THINK - NEED TO LOOK MORE CLOSELY)\ldots ONE IS WW2 PRETTY
SURE\ldots THE OTHER NEED TO LOOK AGAIN - MAYBE AN UNUSUALLY LARGE
LOWERING OF THE RECORD OR SOMETHING?
\end{frame}

\begin{frame}{A ``Self-Exciting'' Model}
\protect\hypertarget{a-self-exciting-model}{}
\begin{block}{Hawkes Processes}
\protect\hypertarget{hawkes-processes}{}
\begin{itemize}
\tightlist
\item
  Let \(H_t\) be the history of events up to time \(t\). The Hawkes
  (1971) model of the conditional intensity is:
\end{itemize}

\[\lambda(t|H_t) = \nu + \sum_{i:t_i<t} g(t - t_i)\] where \(\nu\) is
the background rate of events and g is the ``triggering function''.

\begin{itemize}
\tightlist
\item
  The ``triggering'' function can be further decomposed:
\end{itemize}

\[g = \mu g^*\] where \(g^*\) is a density function known as the
``reproduction kernel'' and \(\mu\) is known as the ``reproduction''
mean.

\begin{itemize}
\tightlist
\item
  A common choice for the ``reproduction kernel'' is the exponential
  density given by:
\end{itemize}

\[g^*(t) = \beta e^{-\beta t} \]
\end{block}
\end{frame}

\begin{frame}{Fitting the model}
\protect\hypertarget{fitting-the-model}{}
Parameter estimates for marathon data (exponential) Hawkes process,
using MLE:

\begin{itemize}
\tightlist
\item
  baseline intensity 0.396
\item
  reproduction mean 0.121
\item
  exponential reproduction function rate 3.91
\end{itemize}

Note the baseline intensity is slightly lower than the constant model
rate estimate of 0.445

The estimated reproduction function is then:

\[g(t) = \mu g^*(t) = \mu \beta e^{-\beta t} \]
\[ = 0.12 * 3.91 e^{-3.91 t} \]
\end{frame}

\begin{frame}{Model implications}
\protect\hypertarget{model-implications}{}
At the instant of the first event (world record), \(t = t_1\) so
\(g(t - t_1 = 0)\) and the reproduction rate is:

\[g(0) = 0.12 * 3.91 e^{-3.91 0} = 0.12 * 3.91 =  0.471 \]

\begin{itemize}
\item
  The rate increases from the baseline rate of 0.396 by this amount at
  the moment of this occurrence
\item
  The rate then decays back to baseline over time (unless a new event
  occurs).
\item
  Each new event ``excites'' the rate to increase and then decay
\end{itemize}
\end{frame}

\begin{frame}{The Intensity Function over Time}
\protect\hypertarget{the-intensity-function-over-time}{}
Below is based on a simulation of the intensity function over a 100 year
period. NOTE: HERE WOULD BE NICE TO SHOW FOR OUR DATA ALTHOUGH MIGHT NOT
GIVE THE FULL PICTURE ANYWAY

\includegraphics{sturdivant_clark_srcos2022_files/figure-beamer/unnamed-chunk-13-1.pdf}
\end{frame}

\begin{frame}{Intensity compared to the constant rate model}
\protect\hypertarget{intensity-compared-to-the-constant-rate-model}{}
\includegraphics{sturdivant_clark_srcos2022_files/figure-beamer/unnamed-chunk-14-1.pdf}
\end{frame}

\begin{frame}{Process as ``Generations''}
\protect\hypertarget{process-as-generations}{}
\includegraphics{sturdivant_clark_srcos2022_files/figure-beamer/unnamed-chunk-15-1.pdf}
\end{frame}

\begin{frame}{The Compensator Function}
\protect\hypertarget{the-compensator-function}{}
NEED TO WORK ON EXPLAINING - BELOW IS THE VERSION TAKING OUT
BASELINE\ldots MAYBE START WITH CONSTANT (CAN DO Poisson MODEL AND THEN
THE BASELINE RATE HERE)

\includegraphics{sturdivant_clark_srcos2022_files/figure-beamer/unnamed-chunk-16-1.pdf}
\end{frame}

\begin{frame}{Residuals}
\protect\hypertarget{residuals}{}
\end{frame}

\begin{frame}{References}
\protect\hypertarget{references}{}
Data source: Wikipedia
(\url{https://en.wikipedia.org/wiki/Marathon_world_record_progression})
scraped August 12, 2022

Poisson process:
\url{https://towardsdatascience.com/the-poisson-distribution-and-poisson-process-explained-4e2cb17d459}

Hawkes, Alan G. 1971. ``Spectra of Some Self-Exciting and Mutually
Exciting Point Processes.'' Biometrika 58 (1): 83--90.
\url{https://doi.org/10.2307/2334319}.

``Hawkesbow'' package\ldots{}
\end{frame}

\end{document}
